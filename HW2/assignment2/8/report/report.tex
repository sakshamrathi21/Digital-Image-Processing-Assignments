\documentclass[12pt]{article}

\usepackage{geometry}
\geometry{a4paper, left=1in, right=1in, top=1in, bottom=1in}
\usepackage{amsmath}
\usepackage{amsmath,amsfonts,amssymb}
\usepackage{graphicx}
\usepackage{enumitem}
\usepackage{titlesec}
\usepackage{fancyhdr}
\usepackage{hyperref}
\usepackage{floatrow}
\usepackage{geometry}
\usepackage{fancyhdr}
\usepackage{empheq}
\usepackage[svgnames]{xcolor}
\usepackage{xpatch}
\usepackage{subcaption} 
\makeatletter
\newcommand{\colorboxed}[1]{\fcolorbox{Black}{White}{\m@th$\displaystyle#1$}}
\xpatchcmd{\@Aboxed}{\boxed}{\colorboxed}{}{}
\makeatother

\title{{\bf CS663 Assignment 2}}
\author{Saksham Rathi, Kavya Gupta, Shravan Srinivasa Raghavan}
\date{September 2024}
\begin{document}
\maketitle
\clearpage
\tableofcontents
\clearpage
\section*{Question 8}
\addcontentsline{toc}{section}{Question 8}
    The results of doing local and global histogram equalisation is as follows:
    
    \begin{figure}[h!]
        \centering
        
        % First row of images (3 images)
        \begin{subfigure}[b]{0.3\textwidth}
            \centering
            \includegraphics[width=\textwidth]{../images/LC1_globalHistEq.png}
            \caption{Global HistEq}
        \end{subfigure}
        \hfill
        \begin{subfigure}[b]{0.3\textwidth}
            \centering
            \includegraphics[width=\textwidth]{../images/LC1_localHistEq_7x7.png}
            \caption{Local HistEq 7x7}
        \end{subfigure}
        \hfill
        \begin{subfigure}[b]{0.3\textwidth}
            \centering
            \includegraphics[width=\textwidth]{../images/LC1_localHistEq_31x31.png}
            \caption{Local HistEq 31x31}
        \end{subfigure}
        
        \vspace{10pt} % Space between the two rows
        
        % Second row of images (3 images)
        \begin{subfigure}[b]{0.3\textwidth}
            \centering
            \includegraphics[width=\textwidth]{../images/LC1_localHistEq_51x51.png}
            \caption{Local HistEq 51x51}
        \end{subfigure}
        \hfill
        \begin{subfigure}[b]{0.3\textwidth}
            \centering
            \includegraphics[width=\textwidth]{../images/LC1_localHistEq_71x71.png}
            \caption{Local HistEq 71x71}
        \end{subfigure}
        \hfill
        \begin{subfigure}[b]{0.3\textwidth}
            \centering
            \includegraphics[width=\textwidth]{../images/LC1.png}
            \caption{Original LC1}
        \end{subfigure}
    
        \caption{Results for LC1.png}
    \end{figure}

    \begin{figure}[h!]
        \centering
        
        % First row of images (3 images)
        \begin{subfigure}[b]{0.3\textwidth}
            \centering
            \includegraphics[width=\textwidth]{../images/LC2_globalHistEq.png}
            \caption{Global HistEq}
        \end{subfigure}
        \hfill
        \begin{subfigure}[b]{0.3\textwidth}
            \centering
            \includegraphics[width=\textwidth]{../images/LC2_localHistEq_7x7.png}
            \caption{Local HistEq 7x7}
        \end{subfigure}
        \hfill
        \begin{subfigure}[b]{0.3\textwidth}
            \centering
            \includegraphics[width=\textwidth]{../images/LC2_localHistEq_31x31.png}
            \caption{Local HistEq 31x31}
        \end{subfigure}
        
        \vspace{10pt} % Space between the two rows
        
        % Second row of images (3 images)
        \begin{subfigure}[b]{0.3\textwidth}
            \centering
            \includegraphics[width=\textwidth]{../images/LC2_localHistEq_51x51.png}
            \caption{Local HistEq 51x51}
        \end{subfigure}
        \hfill
        \begin{subfigure}[b]{0.3\textwidth}
            \centering
            \includegraphics[width=\textwidth]{../images/LC2_localHistEq_71x71.png}
            \caption{Local HistEq 71x71}
        \end{subfigure}
        \hfill
        \begin{subfigure}[b]{0.3\textwidth}
            \centering
            \includegraphics[width=\textwidth]{../images/LC2.jpg}
            \caption{Original LC2}
        \end{subfigure}
    
        \caption{Results for LC2.png}
    \end{figure}

    Upon comparing the global and local histogram equalisation for LC1 the patches outlined in the red boxes
    were found to have \textbf{better contrast} in the local method than in the global method:

    \begin{figure}[ht]
        \centering
        
        % First row of images (3 images)
        \begin{subfigure}[b]{0.4\textwidth}
            \centering
            \includegraphics[width=\textwidth]{../images/LC1_globalHistEq_1.jpeg}
            \caption{Global HistEq}
        \end{subfigure}
        \hfill
        \begin{subfigure}[b]{0.4\textwidth}
            \centering
            \includegraphics[width=\textwidth]{../images/LC1_localHistEq_1.jpeg}
            \caption{Local HistEq}
        \end{subfigure}
        
        \vspace{10pt} % Space between the two rows
        
        % Second row of images (3 images)
        \begin{subfigure}[b]{0.4\textwidth}
            \centering
            \includegraphics[width=\textwidth]{../images/LC1_globalHistEq_2.jpeg}
            \caption{Global HistEq}
        \end{subfigure}
        \hfill
        \begin{subfigure}[b]{0.4\textwidth}
            \centering
            \includegraphics[width=\textwidth]{../images/LC1_localHistEq_2.jpeg}
            \caption{Local HistEq}
        \end{subfigure}
    
        \caption{Contrasts for LC1.png}
    \end{figure}

    Similar results were obtained for the image LC2:

    \begin{figure}[ht]
        \centering
        
        % First row of images (3 images)
        \begin{subfigure}[b]{0.4\textwidth}
            \centering
            \includegraphics[width=\textwidth]{../images/LC2_globalHistEq_1.jpeg}
            \caption{Global HistEq}
        \end{subfigure}
        \hfill
        \begin{subfigure}[b]{0.4\textwidth}
            \centering
            \includegraphics[width=\textwidth]{../images/LC2_localHistEq_1.jpeg}
            \caption{Local HistEq 7x7}
        \end{subfigure}
        
        \vspace{10pt} % Space between the two rows
        
        % Second row of images (3 images)
        \begin{subfigure}[b]{0.4\textwidth}
            \centering
            \includegraphics[width=\textwidth]{../images/LC2_globalHistEq_2.jpeg}
            \caption{Global Histeq}
        \end{subfigure}
        \hfill
        \begin{subfigure}[b]{0.4\textwidth}
            \centering
            \includegraphics[width=\textwidth]{../images/LC2_localHistEq_2.jpeg}
            \caption{Local HistEq 71x71}
        \end{subfigure}
    
        \caption{Contrasts for LC2.jpg}
    \end{figure}
\end{document}