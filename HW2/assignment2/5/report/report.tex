\documentclass[12pt]{article}

\usepackage{geometry}
\geometry{a4paper, left=1in, right=1in, top=1in, bottom=1in}
\usepackage{amsmath}
\usepackage{amsmath,amsfonts,amssymb}
\usepackage{graphicx}
\usepackage{enumitem}
\usepackage{titlesec}
\usepackage{fancyhdr}
\usepackage{hyperref}
\usepackage{floatrow}
\usepackage{geometry}
\usepackage{fancyhdr}
\usepackage{empheq}
\usepackage[svgnames]{xcolor}
\usepackage{xpatch}

\makeatletter
\newcommand{\colorboxed}[1]{\fcolorbox{Black}{White}{\m@th$\displaystyle#1$}}
\xpatchcmd{\@Aboxed}{\boxed}{\colorboxed}{}{}
\makeatother

\title{{\bf CS663 Assignment 2}}
\author{Saksham Rathi, Kavya Gupta, Shravan Srinivasa Raghavan}
\date{September 2024}
\begin{document}
\maketitle
\clearpage
\tableofcontents
\clearpage
\section*{Question 5}
\addcontentsline{toc}{section}{Question 5}

Let us first consider the case of zero-mean Gaussian Filter. We will choose a filter of size $2a+1$ and it is already given that the ramp image has infinite extent. The 1D image has intensities of the form:
\[I(x) = cx + d\]

The result of applying the Gaussian filter on I will be:
\[J(x) = \frac{\sum_{i=-a}^{a}G(i)I(x+i)}{\sum_{i=-a}^{a}G(i)}\]
where $G(i) = e^{-\frac{i^2}{2\sigma^2}}$ (the constant has been skipped because we are normalizing the weights after applying the filter).

So,
\[J(x) = \frac{\sum_{i=-a}^{a}e^{-\frac{i^2}{2\sigma^2}}(c(x+i)+d)}{\sum_{i=-a}^{a}e^{-\frac{i^2}{2\sigma^2}}}\]We can break the numerator sum as follows:
\[\sum_{i=-a}^{a}e^{-\frac{i^2}{2\sigma^2}}(c(x+i)+d) = (cx+d)\sum_{i=-a}^{a}e^{-\frac{i^2}{2\sigma^2}} + c\sum_{i=-a}^{0}i\times e^{-\frac{i^2}{2\sigma^2}} + c\sum_{i=0}^{a}i\times e^{-\frac{i^2}{2\sigma^2}}\]

The second and third terms on the right hand side, will cancel each other, so here is the final expression of the filtered image:
\[J(x) = \frac{(cx+d)\sum_{i=-a}^{a}e^{-\frac{i^2}{2\sigma^2}}}{\sum_{i=-a}^{a}e^{-\frac{i^2}{2\sigma^2}}} = cx+d = I(x)\]


Therefore, we have deduced that the filtered image $J(x)$ is essentially the same as the original unfiltered image.


Let us move to the case of bilateral filter. Again, we will choose a window of size $2a+1$.
\[J(x) = \frac{\sum_{i=-a}^a G_{\sigma_s}(|(x+i)-x|)G_{\sigma_r}(|I(x+i)-I(x)|)I(x+i)}{\sum_{i=-a}^aG_{\sigma_s}(|(x+i)-x|)G_{\sigma_r}(|I(x+i)-I(x)|)}\]

\[J(x) = \frac{\sum_{i=-a}^a G_{\sigma_s}(|i|)G_{\sigma_r}(|ci|)I(x+i)}{\sum_{i=-a}^aG_{\sigma_s}(|i|)G_{\sigma_r}(|ci|)}\]

\[J(x) = \frac{\sum_{i=-a}^a e^{-\frac{i^2}{\sigma_s^2}} e^{-\frac{(ci)^2}{\sigma_s^2}}(c(x+i)+d)}{\sum_{i=-a}^a e^{-\frac{i^2}{\sigma_s^2}} e^{-\frac{(ci)^2}{\sigma_s^2}}}\]
Constants from the Gaussian PDF have been removed because of normalization in the denominator.


Similar to the previous case, we will break the numerator sum as follows:
\[\sum_{i=-a}^a e^{-\frac{i^2}{\sigma_s^2}} e^{-\frac{(ci)^2}{\sigma_s^2}}(c(x+i)+d) = (cx+d)\sum_{i=-a}^a e^{-\frac{i^2}{\sigma_s^2}} e^{-\frac{(ci)^2}{\sigma_s^2}} + c\sum_{i=-a}^{0}i\times e^{-\frac{i^2}{\sigma_s^2}} e^{-\frac{(ci)^2}{\sigma_s^2}} + c\sum_{i=0}^{a}i\times e^{-\frac{i^2}{\sigma_s^2}} e^{-\frac{(ci)^2}{\sigma_s^2}}\]

The second and the third terms from the sum will cancel each other, so here is the final expression of the filtered image:

\[J(x) = \frac{(cx+d)\sum_{i=-a}^a e^{-\frac{i^2}{\sigma_s^2}} e^{-\frac{(ci)^2}{\sigma_s^2}}}{\sum_{i=-a}^a e^{-\frac{i^2}{\sigma_s^2}} e^{-\frac{(ci)^2}{\sigma_s^2}}} = cx+d = I(x)\]


Therefore, we have deduced that the output image $J(x)$ (after applying bilateral filter) is essentially the same as the original unfiltered image.


\end{document}